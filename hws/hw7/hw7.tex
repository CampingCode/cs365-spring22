\documentclass[12pt]{article}
\usepackage[margin=1in]{geometry} 
\usepackage{amsmath}
\usepackage{tcolorbox}
\usepackage{amssymb}
\usepackage{amsthm}
\usepackage{lastpage}
\usepackage{fancyhdr}
\usepackage{accents}
\pagestyle{fancy}
\setlength{\headheight}{40pt}


\newenvironment{solution}
  {\renewcommand\qedsymbol{$\blacksquare$}
  \begin{proof}[Solution]}
  {\end{proof}}
\renewcommand\qedsymbol{$\blacksquare$}

\newcommand{\ubar}[1]{\underaccent{\bar}{#1}} % add packages, settings, and declarations in settings.tex
\usepackage{url}
\begin{document}

\lhead{Prof. C.E. Tsourakakis} 
\rhead{CS365 Spring '22 \\ Foundations of Data Science \\ Assignment 7} 
\cfoot{\thepage\ of \pageref{LastPage}}


\section*{Instructions}
\framebox{%
	\begin{minipage}{0.9\linewidth}
		\begin{itemize}
			\item The homework is due on \underline{{\bf Friday 4/1 at 5pm ET}}.  
			\item There are 3 problems. The last problem is on Git, and it is a programming assignment.
			\item No extension will be provided, unless for serious documented reasons.
			\item {\bf Start early!}
			\item Study the material taught in class, and feel free to do so in small groups, but the solutions should be a product of your own work. 
			%\item For any given problem, the points per sub-question are equally distributed {\it unless otherwise told}. For example, Problems 2.1, 2.2, 2.3  are worth $\frac{30}{3}=10$ points each. Problem 2.2(i) and (ii) similarly are worth 5 points each. However, Problem 3 has unequal distribution of points that is explicitly given. 
			\item This  is not a multiple choice homework;   reasoning, and mathematical proofs are required before giving your final answer.
		\end{itemize}
\end{minipage}}


\section{Projections [30 points]}   
	Let $\pi_b(x)$ be the orthogonal projection of $x\in \mathbb{R}^n$ on the subspace $U=span(b)$ where $b\in \mathbb{R}^n$. 
\begin{itemize}

	\item[(a)] 	Prove that $\pi_b(x)$  is the closest vector to $x$ on $U$.  
 
	
	\item[(b)] Prove that the Euclidean length of $\pi_b(x)$  is less than or equal to that of  $x$. 
 
	\item[(c)] Can two orthogonal vectors be linearly dependent? Give an answer with a proof. 
 
	\item[(d)] Transform the basis $B= \{ v_1=(4,2), v_2=(1,2) \}$ of $\mathbb{R}^2$ into an orthonormal basis whose first basis
	vector is in the span of $v_1$.
 
\end{itemize}
   
\section{A special matrix-vector multiplication [20 points]} 


Let $u \in \mathbb{R}^n$ be a fixed vector. Let $U = uu^T$. Show that maximizing $x^T U(\vec{1}-x)$ over all binary vectors $x \in \{0,1\}^n$  
is equivalent to partitioning the coordinates of $u$ into two subsets wheret he sum of the elements in both subsets are as equal as possible. Here $1$ represent the all-ones vectors $\vec{1}= \underbrace{[1,1,...,1]^T}_{n \text{~coordinates}}$.
 

\section{$F_0, F_1$ estimation : Programming Assignment [50 points] } 

Please see our Github repo for this problem.

 
\end{document}
