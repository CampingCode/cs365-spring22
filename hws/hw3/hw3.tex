\documentclass[12pt]{article}
\usepackage[margin=1in]{geometry} 
\usepackage{amsmath}
\usepackage{tcolorbox}
\usepackage{amssymb}
\usepackage{amsthm}
\usepackage{lastpage}
\usepackage{fancyhdr}
\usepackage{accents}
\pagestyle{fancy}
\setlength{\headheight}{40pt}


\newenvironment{solution}
  {\renewcommand\qedsymbol{$\blacksquare$}
  \begin{proof}[Solution]}
  {\end{proof}}
\renewcommand\qedsymbol{$\blacksquare$}

\newcommand{\ubar}[1]{\underaccent{\bar}{#1}} % add packages, settings, and declarations in settings.tex
\usepackage{url}
\usepackage{amsmath,amssymb,amsthm}


\newcommand{\Prob}[1]{{{\bf{Pr}}\left[{#1}\right]}}
\newcommand{\Mean}[1]{{\mathbb E}\left[{#1}\right]}
\newcommand{\Var}[1]{{\mathbb Var}\left[{#1}\right]}


\begin{document}

\lhead{Prof. C.E. Tsourakakis} 
\rhead{CS365 Spring '22 \\ Foundations of Data Science \\ Assignment 3} 
\cfoot{\thepage\ of \pageref{LastPage}}


\section*{Instructions}
\framebox{%
	\begin{minipage}{0.9\linewidth}
		\begin{itemize}
			\item The homework is due on \underline{{\bf Friday 2/18 at 5pm ET}}.  
			\item There are 3 problems in total.
			\item No extension will be provided, unless for serious documented reasons.
			\item {\bf Start early!}
			\item Study the material taught in class, and feel free to do so in small groups, but the solutions should be a product of your own work. 
			\item This  is not a multiple choice homework;   reasoning, and mathematical proofs are required before giving your final answer.
		\end{itemize}
\end{minipage}}

\section{Short exercises [15 points]} 

\begin{enumerate}
	\item [(a)] [5 pts] Prove that $1+x \leq e^x$ for all real $x$. For what range of $x$ is $1+x\approx e^x$ within 0.01. 
\smallskip 
	\item[(b)] [2.5pts] Compute for $n=1,\ldots,10$, $n!$ exactly, and using Stirling's approximation formula $n! \approx \sqrt{2 \pi n}(\frac{n}{e})^n$. What do you observe? Present your findings as a 2-column table. 
\smallskip 
		\item[(c)] [7.5pts] Give an example of a random variable for which Chebyshev's inequality is tight, namely the inequality holds as equality. 
\end{enumerate}   


\section{To Handshake or Not?[25 points]} 

Suppose $n$ people walk into a party. Due to covid-19, each pair $\{i,j\}$ shakes hands with probability only $\frac{1}{10}$. Prove that {\it almost surely} {\bf every} person from that party shook hands in the range $[ (1-\epsilon)\frac{n}{10}, (1+\epsilon)\frac{n}{10}]$. 


\section{Bayes hits again [60 points]}

This exercise is given in the form of Jupyter notebook in our Github page \url{https://github.com/tsourolampis/cs365-spring22} under the HW directory.

 
 

\end{document}
